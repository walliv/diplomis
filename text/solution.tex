\chapter{PCI Express}

The \ac{PCIe} is the interconnect standard that emerged in 2003 from the \ac{PCI} standard. The
\ac{PCIe} addressed the drawbacks of the \ac{PCI} that became increasingly challenging when more
devices and more throughput were required. These include:

\begin{itemize}
        \item Shared-bus topology that, inspite of its simplicity, was rendered unscalable since with
        increasing amount of devices, the bus gets longer which weakens the signal for the farthest
        device.
        \item Common clock which does not fit well with parallel data transmission model especially
        when lane skew takes place, meaning the bits of data travel unevenly fast over the wires.
        \item Half-duplex communication allows only one device to communicate at one point at a
        time.
        \item Reflected-wave signaling which was used to reduce the power consumption on the bus.
        This uses an interference of the sent wave from the master (which had driven the signal to
        the lower voltage than needed for the digital circuits on the bus) and its reflection that
        originated on the end of a bus (there is no termination on the bus' end). By the addition of
        these two waves, the signal reaches the required voltage level for the digital circuits (W:
        cite PCIe book).
\end{itemize}

% TODO: Add picture of PCI topology with some functions and one PCI-to-PCI bridge.

The PCI protocol did not keep pace with increasing demand of computer peripheral communication
structures and needed to be changed. The PCI Express protocol significantly changed the strategy for
both the communication topology and the physical properties of the interconnect. 

The basic architecture is shown in Fig. XXX (host topology). Connections are implemented as
point-to-point links between ports on different device types organized in a tree. The center of the
topology is the \ac{RC}, which provides \emph{Root Ports}. Each device communicating over the
infrastructure (such as today's GPU/FPGA accelerators) is called \emph{Function} and is the major
source of traffic within the system\footnotemark. Each Function contains an \emph{Upstream port} for exchanging
data. Since the number of Root Ports is limited by the processor's interconnect, fan-out is provided
by \emph{PCIe Switches}, which supply multiple \emph{Downstream ports} and a single Upstream
port. Each port supports full-duplex communication and introduces asynchronous clocking model.
Using this model significantly improves the possible throughput of the interconnect where clock gets
reconstructed from the patterns in the incoming data on the receiving port. Using Switches, the
number of devices can be increased arbitrarily, making PCIe a switched network (W: cite PCIe spec).
Although the PCIe specification introduces many advanced features, this thesis presents only the
necessary ones for its purposes. 

\footnotetext{
    The PCIe specification (with concordance to the PCI) defines three System elements to identify
    its devices, namely \emph{Bus}, \emph{Device} and \emph{Function}. Every point-to-point
    connection in the topology makes a separate Bus. The Device identifier stays as a remnant of the PCI
    standard and every Bus contains only one. Within each Device, one or multiple Functions can be
    present. The Function is basically and adressable entity in the configuration space where
    \emph{Endpoint} is the most important type of a Function. Using the term `device' in this thesis
    will refer to a single-function device in the PCIe topology.
  }

% NOTE: Maybe mention BDF identifier if necessary

\section{Communication model}

\section{Link configuration}

\section{Device configuration}

\section{Peer-to-peer transfer}
